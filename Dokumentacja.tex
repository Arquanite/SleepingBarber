\documentclass[12pt,a4paper]{article}
\usepackage[margin=0.5in]{geometry}

\usepackage{polski}
\usepackage[utf8]{inputenc}

\usepackage{graphicx}
\usepackage{float}
\usepackage{setspace} %texlive-latex-recommended
\usepackage{enumitem}
\usepackage{tabularx}
\usepackage{listings}
\usepackage{xcolor}
\usepackage{soul}

\setenumerate[1]{label=•}

\newcommand{\imgz}{0.44\linewidth}
\newcommand{\numerzajec}{2}
\newcommand{\tematzajec}{Problem śpiącego fryzjera.}
\newcommand{\datazajec}{8 czerwca 2017}

\definecolor{codegreen}{rgb}{0,0.6,0}
\definecolor{codegray}{rgb}{0.5,0.5,0.5}
\definecolor{backcolour}{rgb}{0.96,0.96,0.96}	

\renewcommand\lstlistingname{Fragment kodu}
\lstdefinestyle{customc}{
  	backgroundcolor=\color{backcolour},   
    commentstyle=\color{codegreen},
    keywordstyle=\color{blue},
    numberstyle=\scriptsize\color{codegray},
    stringstyle=\color{codegreen},
    basicstyle=\footnotesize,
    breakatwhitespace=false,         
    breaklines=true,                 
    captionpos=b,                    
    keepspaces=true,                 
    numbers=left,                    
    numbersep=5pt,                  
    showspaces=false,                
    showstringspaces=false,
    showtabs=false,                  
    tabsize=4,
}
\lstset{style=customc, language=C, escapeinside={(*@}{@*)}}

\begin{document}
	\thispagestyle{empty}
	\begin{center}
		\vspace*{1.6cm}
		\includegraphics[width=0.55\linewidth]{pblogo.png}	\\
		\vspace{0.5cm}
		\large
		\textsf{\textbf{Dokumentacja}} \\
		\vspace{0.5cm}
		\textsf{\textbf{\textit{Systemy operacyjne}}}	\\
		\vspace{1cm}
		\textsf{Projekt numer: \textbf{\numerzajec}}	\\
		\vspace{0.5cm}
		\textsf{Temat: \textbf{\tematzajec}}
	\end{center}

	\vspace{2cm}
	\begin{tabular}{rl}
        Wykonujący ćwiczenie: & \textbf{Adrian Kryński} \\
                              &\textbf{Tobiasz Dzieszkowski} \\        									
	\end{tabular}

	\vspace{3.5cm}		

	\begin{minipage}{0.45\linewidth}
		\large
		\begin{spacing}{1.5}
		Studia dzienne \\
		Kierunek: Informatyka \\
		Semestr: IV \\
		\end{spacing}
	\end{minipage}
	\begin{minipage}[t]{0.5\linewidth}
		\large
		\begin{spacing}{1.2}
		Grupa zajęciowa: PS4
		\end{spacing}
	\end{minipage}
	
	Prowadzący ćwiczenie: \textbf{mgr inż. Daniel Reska} \\
	
	\begin{flushright}
		\begin{minipage}[t]{0.3\linewidth}
			\centering
			................. \\
			\small OCENA
		\end{minipage}
	\end{flushright}
	
	\begin{minipage}[t]{0.4\linewidth}
		\centering
		Data zajęć \\
		\small \datazajec
	\end{minipage}
	
	\begin{flushright}
		\begin{minipage}[t]{0.5\linewidth}
			\centering
			............................................. \\
			\small \textsf{Data i podpis prowadzącego}
		\end{minipage}
	\end{flushright}
	\pagebreak
	%%%%%%%%%%%%%%%%%%%%%%%%%%%%%%%%%%%%%%%%%%%%%%%%%
	
	\section{Wstęp}
	Celem projektu było napisanie programów koordynujących pracę
	gabinetu z problemu \texttt{śpiącego fryzjera}:
	\begin{enumerate}[label=\alph*)]
		\item wykorzystując zmienne warunkowe
		\item bez wykorzystania zmiennych warunkowych (tylko 
			mutexy/semafory)
	\end{enumerate}
	
	Stworzyliśmy programy implementujące obydwa rozwiązania (łącznie
	34 punkty).
	
	\section{Opcje programu}
	Program udostępnia kilka opcji wspomagających obserwację jego
	działania. Posiada on wartości \texttt{verbose} (domyślnie na 1), 
	\texttt{debug} (domyślnie na 0), \texttt{haircut-time} (domyślnie
	3000ms), oraz \texttt{waitroom-size} (domyślnie na 4). \\
	Im wyższy poziom \texttt{verbose}, tym bardziej szczegółowe 
	informacje na temat działania programu uzyskamy. \\
	Czas strzyżenia (\texttt{haircut-time}) ustawiany jest w 
	milisekundach, zaś wielkość poczekalni (\texttt{waitroom-size}) w
	liczbie osób.
	\begin{enumerate}
		\item \texttt{--debug} - zmienia wartość \texttt{debug} na 1 - 
			tradycyjny debug (np. \texttt{Res:2 WRomm: 5/10 [in: 4]})
		\item \texttt{--debug-experimental} - zmienia wartość
			\texttt{debug} na 2 (eksperymentalny - pływająca ramka)
		\item \texttt{--no-verbose} - zmienia wartość \texttt{verbose}
			na 0 (wyłącza)
		\item \texttt{--very-verbose} - zmienia wartość \texttt{verbose}
			na 2
		\item \texttt{--haircut-time}[time] - zmienia czas strzyżenia na
			\texttt{time}
		\item \texttt{--waitroom-size}[count] - zmienia wielkość
			poczekalni na \texttt{count}
	\end{enumerate}
	
	\section{Wersja wykorzystująca zmienne warunkowe}
	\subsection{Kolejka i zmienne warunkowe}
	
	\begin{center}
		\begin{minipage}{0.8\linewidth}
\begin{lstlisting}[caption = Struktura reprezentująca kolejkę.]
typedef struct {
int id;
pthread_cond_t *cond;
} thread_data;

typedef struct thread_queue thread_queue;

struct thread_queue {
    thread_data data;
    thread_queue *next;
};
\end{lstlisting}
		\end{minipage}
	\end{center}
		
	
	Wersja wykorzystująca zmienne warunkowe posiada kolejkę 
	\texttt{FIFO}, która zapewnia obsługę klientów (wątków) w kolejności 
	\texttt{First Come First Served}.
			
	Każdy element w kolejce zawiera identyfikator (\texttt{id}) oraz
	zmienną warunkową (\texttt{cond}) używaną do wybudzania wątku.
	
	\subsection{Mutexy}
	
	\begin{center}
		\begin{minipage}{0.8\linewidth}
\begin{lstlisting}[caption = Deklaracje mutexów.]
// Mutex zabezpieczajacy poczekalnie
pthread_mutex_t waitroom = PTHREAD_MUTEX_INITIALIZER;

// Mutex zabezpieczajacy kase biletowa
pthread_mutex_t number = PTHREAD_MUTEX_INITIALIZER;

// Mutex zabezpieczajacy gabinet
pthread_mutex_t haircut = PTHREAD_MUTEX_INITIALIZER;
\end{lstlisting}
		\end{minipage}
	\end{center}
	
	W programie są wykorzystywane trzy mutexy:
	\begin{enumerate}
		\item \texttt{waitroom} - zapewnia że tylko jeden klient na raz
			może zmieniać stan poczekalni
		\item \texttt{number} - zabezpiecza kasę biletową dając do niej
			dostęp jednemu klientowi na raz
		\item \texttt{haircut} - zapewnia że tylko jeden klient
			jednocześnie będzie w stanie wejść do gabinetu
	\end{enumerate}
	
	\subsection{Semafory}
	
	\begin{center}
		\begin{minipage}{0.8\linewidth}
\begin{lstlisting}[caption = Deklaracje semaforów.]
sem_t waiting_clients; // Semafor zliczajacy oczekujacych klientow
sem_t haircut_done; // Semafor informujacy o zakonczeniu strzyzenia
sem_t waitroom_seats; // Semafor zliczajacy wolne miejsca w poczekalni
\end{lstlisting}
		\end{minipage}
	\end{center}
	
	W programie są wykorzystywane trzy semafory:
	\begin{enumerate}
		\item \texttt{waiting{\_}clients} - semafor odpowiedzialny za
			zliczanie klientów oczekujących w poczekalni na strzyżenie
		\item \texttt{haircut{\_}done} - semafor odpowiedzialny za
			określenie kiedy strzyżenie zostało zakończone
		\item \texttt{waitroom{\_}seats} - semafor odpowiedzialny za
			zliczanie wolnych miejsc w poczekalni
	\end{enumerate}
	
	\subsection{Fryzjer}	
	
	\begin{center}
		\begin{minipage}{0.82\linewidth}
\begin{lstlisting}[caption = Funkcja fryzjera.]
void barber(){
    while(true){
        if(verbose > 0) printf("Barber is waiting for customers...\n");
        sem_wait(&waiting_clients); // Oczekiwanie na klientow
        // Sprawdzenie czy klienci zajeli swoje miejsca w poczekalni
        pthread_mutex_lock(&waitroom);
        if(verbose > 1) printf("Barber wakes next customer\n");
        clients = queue_dequeue(clients); // Uzyskanie danych klienta
        print_queue(clients, "Clients");
        if(verbose > 1) printf("Barber is doing a haircut\n");
        pthread_cond_signal(clients->data.cond); // Obudzenie klienta
        // Odblokowanie poczekalni aby klient mogl wejsc do gabinetu
        pthread_mutex_unlock(&waitroom);
        sem_wait(&haircut_done); // Oczekiwanie na zakonczenie strzyzenia
        if(verbose > 1) printf("Barber has done a haircut\n");
    }
}
\end{lstlisting}
		\end{minipage}
	\end{center}
	
	Fryzjer na samym początku sprawdza czy w poczekalni znajdują się
	jacyś klienci (\texttt{linia 4)}. Jeżeli nie ma - idzie spać do 
	czasu aż nie zjawi się nowy klient. Jeżeli w poczekalni ktoś jest 
	fryzjer upewnia się, że wszyscy oczekujący klienci zajęli swoje 
	miejsca w poczekalni (\texttt{linia 6}) - zabezpiecza to przed
	wyścigiem klientów, który może prowadzić do zakleszczenia.
	
	Fryzjer bierze dane pierwszego klienta z kolejki (\texttt{linia 8}),
	a następnie go budzi używając zmiennej warunkowej 
	(\texttt{linia 11}). 
	
	Następnie wraca do gabinetu i przebywa tam aż
	do momentu zakończenia strzyżenia (\texttt{linie 13-14}). Zapobiega
	to zawołaniu do gabinetu wielu klientów jednocześnie.
	
	\subsection{Klient}
	
	\begin{center}
		\begin{minipage}{0.82\linewidth}
\begin{lstlisting}[caption = Funkcja klienta 1/3.]
void client(pthread_cond_t *cond){
    // Klient wchodzi do poczekalni
    pthread_mutex_lock(&waitroom);
    // Klient bierze numerek (id)
    int id = take_number();
    if(verbose > 0) printf("New client takes number: %d\n", id);
    // Klient probuje usiasc
    if(sem_trywait(&waitroom_seats) != 0){
        // Klient rezygnuje
        if(verbose > 0) printf("There's no free seats! Client %d goes out\n", id);
        resign_count++;
        // Zwiekszenie licznika i dodanie do kolejki zrezygnowanych
        queue_enqueue(resigned, cond, id);
        print_queue(resigned, "Resigned");
        print_stats();
        pthread_mutex_unlock(&waitroom);
        return;
    }
\end{lstlisting}
		\end{minipage}
	\end{center}
	
	Klient przed wejściem do poczekalni czeka aż poprzedni klient
	zajmie swoje miejsce (usiądzie, pójdzie do gabinetu lub opuści salon
	fryzjerski - \texttt{linia 3}). Zapobiega to wyścigom do kasy i na
	krzesło w poczekalni. Po wejściu do poczekalni kieruje się do kasy 
	biletowej, skąd pobiera numerek (przydzielenie \texttt{id} -
	\texttt{linia 5}).
	
	Następnie klient sprawdza czy w poczekalni są wolne miejsca, jeżeli
	nie ma to rezygnuje ze strzyżenia i wychodzi z poczekalni
	(\texttt{linie 8-18}).
	
	\begin{center}
		\begin{minipage}{0.8\linewidth}
\begin{lstlisting}[caption = Funkcja klienta 2/3.]
// Dodanie klienta do kolejki
    queue_enqueue(clients, cond, id);
    print_queue(clients, "Clients");
    // Zwiekszenie liczby oczekujacych klientow
    sem_post(&waiting_clients);
    print_stats();
    if(verbose > 1) printf("Client %d waits for haircut\n", id);
    // Klient czeka na swoja kolej
    pthread_cond_wait(cond, &waitroom);
    // Klient zwalnia miejsce w poczekalni
    sem_post(&waitroom_seats);
    // Klient opuszcza poczekalnie
    pthread_mutex_unlock(&waitroom);
\end{lstlisting}
		\end{minipage}
	\end{center}
	
	Jeżeli w poczekalni są wolne miejsca, klient ustawia się w kolejce
	(zajmuje wolne krzesło - \texttt{linie 2 i 5}), a następnie oczekuje
	na swoją kolej (\texttt{linia 9}). Kiedy klient zostanie zawołany
	przez fryzjera zwalnia krzesło w poczekalni (\texttt{linia 11}) i
	opuszcza ją (\texttt{linia 13}).
	
	\begin{center}
		\begin{minipage}{0.8\linewidth}
\begin{lstlisting}[caption = Funkcja klienta 3/3.]
// Wejscie do gabinetu
    pthread_mutex_lock(&haircut);
    // Ustawienie id aktualnie obslugiwanego klienta
    served_client = id;
    print_stats();
    if(verbose > 0) printf("Client %d is getting a haircut\n", id);
    usleep(1000*haircut_time);
    // Zakonczenie strzyzenia
    sem_post(&haircut_done);
    served_client = -1;
    print_stats();
    // Opuszczenie poczekalni
    pthread_mutex_unlock(&haircut);
}
\end{lstlisting}
		\end{minipage}
	\end{center}
	
	Klient wchodzi do gabinetu (\texttt{linia 2}), gdzie zostaje
	ostrzyżony, po czym informuje fryzjera o zakończeniu wizyty
	(\texttt{linia 9}) - zapobiega to zawołaniu przez fryzjera kolejnego
	klienta zanim aktualnie obsługiwany nie zakończy strzyżenia - a 
	następnie opuszcza gabinet (\texttt{linia 13}).
	
		\section{Wersja bez zmiennych warunkowych}
	
	\begin{lstlisting}[label = kod, caption = rak]
#include <stdio.h>
#define N 10
/* Block
* comment */

int main(){
	int i;
	(*@ \hl{cout<<"Hello world";} @*)
	for(i=0; i<5; i++){
		cout<<i<<endl;
	}
	return 0;
}

	\end{lstlisting}
	
	
	
\end{document}
